\section{Related Works}

\paragraph{Human Simulation and Persona Modeling} AI agents demonstrate remarkable progress in simulating human behavior. Park et al.~\cite{Park2024GenerativeAgentSimulations,Park2025SimulatingHumanBehavior} show generative agents replicate survey responses with 85\% accuracy compared to human self-consistency, while Park et al.~\cite{Park2023GenerativeAgents} introduce architectures combining memory, reflection, and planning for believable behavior including emergent social interactions. Persona modeling has evolved from descriptive sentences~\cite{Sutcliffe2023SurveyPersonalityPersonaProfile} to dynamic systems with internal states and emotions~\cite{WangChiu2023HumanoidAgents,Feng2025EmotionallyIntelligent}, though challenges remain including systematic biases~\cite{Li2025LLMGeneratedPersona} and personality generation difficulties~\cite{Molchanova2025ExploringPersonality}. Ahmad et al.~\cite{Ahmad2025SimulatingUserDiversity} emphasize behavioral diversity in user simulation, while Sun et al.~\cite{Sun2022MetaphoricalUserSimulators} explore metaphorical approaches. Xie et al.~\cite{Xie2024HumanSimulacra} demonstrate multi-agent cognitive mechanisms producing personified responses aligned with target characters, while Castricato et al.~\cite{Castricato2024PERSONA} and Ge et al.~\cite{Ge2024PersonaHub} procedurally generate diverse personas from demographic data. Chu et al.~\cite{Chu2024CohesiveConversations} highlight conversational coherence for maintaining persona consistency across multi-turn interactions.

\paragraph{Multi-Agent Orchestration and Coordination} Decomposing complex tasks into specialized agents has emerged as a powerful paradigm for managing system complexity. Lee et al.~\cite{Lee2024OrchestraLLM} and Zhang et al.~\cite{Zhang2025AgentOrchestra} demonstrate efficient orchestration through routing frameworks that strategically select between models, reducing computational costs by 50\% while improving performance. Dang et al.~\cite{Dang2025MultiAgentCollaboration} and Tran et al.~\cite{Tran2025MultiAgentCollaboration} introduce dynamic orchestration with centralized coordinators trained via reinforcement learning, evolving from static to adaptive structures. Bernard and Balog~\cite{Bernard2024FormalCharacterizationUserSimulation} formalize dialogue state and action spaces for conversational systems, while Balog and Zhai~\cite{Balog2025UserSimulationEraGenerativeAI} and Davidson et al.~\cite{Davidson2023UserSimulationLLM} emphasize combining LLMs with additional components to capture cognitive processes. Raza et al.~\cite{Raza2024TRiSM} introduce metrics like Component Synergy Score and Tool Utilization Efficacy for quantifying collaboration quality, while Shu et al.~\cite{Shu2024EffectiveMultiAgent} demonstrate 90\% goal success rates in multi-agent collaboration, highlighting the importance of structured protocols.

\paragraph{Cognitive Architectures and State Management} Cognitive science provides crucial insights for agent design. Sumers et al.~\cite{Sumers2024CoALA} propose CoALA, drawing from symbolic AI to organize agents with modular memory components and structured action spaces mirroring human cognitive processes. Hu and Ying~\cite{Hu2025UnifiedMindModel} present architectures based on Global Workspace Theory incorporating perception, planning, reasoning, memory, and motivation components. For dialogue systems, Niu et al.~\cite{Niu2024EnhancingDST} and Xu et al.~\cite{Xu2024ChainOfThoughtDST} use LLM-backed agents with chain-of-thought reasoning to generate annotated dialogues for state tracking, while Levi and Kadar~\cite{Levi2025IntellAgent} introduce graph-based modeling for multi-turn dialogues with policy constraints. Mehri et al.~\cite{Mehri2025GoalAlignment} emphasize goal alignment ensuring state tracking remains consistent with user objectives. These architectures emphasize separation between working memory (state tracking) and behavioral planning (motivation systems), validating specialized agent approaches for complex dialogue domains~\cite{Rastogi2019SchemaGuidedDialogue,Yi2024MultiTurnDialogueSurvey,Mo2024HierTOD}.

\paragraph{Synthetic Data Generation and Evaluation} Agent-based systems require sophisticated evaluation methodologies beyond traditional metrics. Zhuge et al.~\cite{Zhuge2024AgentAsAJudge} show Agent-as-a-Judge achieves 90\% alignment with human consensus while reducing evaluation costs by 97\%, dramatically outperforming LLM-as-a-Judge approaches. For synthetic data generation, Suresh et al.~\cite{Suresh2024DiaSynth} use Chain of Thought reasoning to generate dialogues achieving 90.48\% of in-domain data performance, while Devanathan et al.~\cite{Devanathan2025WhySynthetic} introduce 18 linguistically grounded metrics revealing deficits in sentiment and behavioral realism. Evaluation frameworks must address task completion, output quality, consistency, and robustness~\cite{Mohammadi2025EvaluationBenchmarking}, with Zhu et al.~\cite{Zhu2025EvolutionaryEvaluationLLMAgents,Zhu2025AgenticBenchmarks} emphasizing preventing trivial shortcuts and ensuring agents genuinely leverage persona and state understanding. Wang et al.~\cite{Wang2025SurveyLLMAgents} identify critical limitations in role-playing, alignment, and knowledge boundaries that multi-agent approaches can address.